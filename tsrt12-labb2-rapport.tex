%______________________________________________________
%
%   Vid funderingar titta längst ned i denna fil,
%   eller skicka ett mail
%______________________________________________________
%

\documentclass[a4wide]{article}
\usepackage{fixltx2e}
\usepackage[utf8]{inputenc}
\usepackage{graphicx}
\usepackage{sidecap}
\usepackage{fancyhdr}
\usepackage[margin=1.5in]{geometry}
\usepackage{amssymb,amsmath}
\usepackage[euler]{textgreek}
\usepackage[swedish]{babel}

\setlength{\headheight}{21pt}


% Ändra de rader som behöver ändras
\def\inst{Systemvetenskap}
\def\typeofdoc{Laborationsrapport}
\def\course{Reglerteknik}
\def\pretitle{Laboration 2}
\def\title{Modellbaserad reglering av dubbeltankar}
\def\name{Peter Arvidsson, Sara Svensson}
%\def\username{c00abc}
%\def\email{\username{}@cs.umu.se}
\def\email{petarXXX@student.liu.se, sasv839@student.liu.se}
%\def\path{edu/KURS/lab1}
%\def\graders{ALLA HANDLEDARNAS NAMN}

% Här börjar själva dokumentet
\begin{document}

	% skapar framsidan (om den inte duger: gör helt enkelt en egen)
	\begin{titlepage}
		\thispagestyle{empty}
		\begin{large}
			\begin{tabular}{@{}p{\textwidth}@{}}
				\textbf{Linköpings universitet \hfill \today} \\
				\textbf{Institutionen för \inst} \\
				\textbf{\typeofdoc} \\
			\end{tabular}
		\end{large}
		\vspace{10mm}
		\begin{center}
		~\\
		~\\
		~\\
		~\\
		~\\
		~\\
			\LARGE{\pretitle} \\
			\huge{\textbf{\title}}\\
			\vspace{10mm}
			\LARGE{\course} \\
			\vspace{15mm}
			\begin{large}
				\begin{tabular}{ll}
					\textbf{Namn} & \name \\
					\textbf{E-mail} & \texttt{\email} \\
					%\textbf{Sökväg} & \texttt{\fullpath} \\
				\end{tabular}
			\end{large}
			%\large{\textbf{Handledare}}\\
			\mbox{\large{\graders}}
		\end{center}
		~\\
		~\\
		~\\
		~\\
		~\\
		I denna rapport har en regulator tagits fram för ett system med dubbla vattentankar vars stegsvar skulle uppfylla vissa kriterium. Vattentankarna var seriekopplade på höjden på sådant sätt att utflödet på den övre tanken var inflödet på den undre. Inflödet för den övre tanken styrdes av en pump. 
		 Innan regulatorn kunde tas fram behövdes överföringsfunktionens stigtid och konstant för det öppna systemet undersökas. Detta gjordes genom mätningar på övre och undre tankens stegsvar var för sig. Regulatorn togs sedan fram på liknande vis, men där överslänget, stigtiden eller det stationära felet minskades ifall de gav för stora värden på hela systemets stegsvar. 
	%\vfill
	\end{titlepage}
	
\end{document}

\begin{document}

% Innehåll ------------------------------
\newpage
\thispagestyle{empty}
\tableofcontents
\newpage



\listoffigures
~\\

\listoftables
\newpage


%header ---------------------------------
\pagestyle{fancy}

\fancyhead{} % clear all header fields
\fancyhead[L]{TSRT12\slshape}
\fancyhead[R]{\today \slshape}

\fancyfoot{} % clear all footer fields
\fancyfoot[L,R]{\thepage}
\fancyfoot[L]{Modellbaserad reglering av dubbeltankar}

%slut på header ---------------------------------


\section{Inledning}
För att få erfarenhet av att ta fram en modellbaserad regulator, fick vi i uppdrrag att experimenteriellt ta fram en regulator för ett system med dubbla vattentankar. Till vår hjälp fick vi ett tanksystem i form av två seriekopplade tankar, en pump, styr- och märkort, samt ett spänningsaggregat. Vi fick också programmen \texttt{LeadLag} och \texttt{Simulation} som användes för att styra respektive simulera dubbeltanken och regulatorn. För att lösa problemet skickade vi som insignal in ett enhetssteg i systemanordningen och avläste stigtiden, överslänget och det stationära felet ur stegsvaret. Regulatorn skulle uppfylla följande krav:
	\begin{itemize}
		\item Stigtid T_{r} \leq 5 sekunder,
		\item Översläng M \leq 10 \%,	
		\item Stationärt fel e_{0} \leq 5 \%. 
	\end{itemize}
\\\\Resterande rapport kommer att ta upp en mer grundläggande beskrivning utav våra upptäckter i laborationen i form av experimentuppställning, utförande, felanalys, diskussion och slutsats.


% -------------- Experimentuppställning-----------------


\section{Experimentuppställning}
Till expeimentet användes:
		\begin{itemize}
			\item Vattentanksystem bestående av två stycken vattentanker, em pump och två vattennivåsensorer.		
			\item Programmet \texttt{LeadLag}
			\item Programmet \texttt{Simulation}
			\item Boken \textsl{Re3glerteknik grundläggande teori} av Torkel Glad och Lennart Ljung
		\end{itemize}



% ------------- utförande -----------------------


\section{Utförande}


\begin{equation}
G\textsubscript{dubbel}(s) = \frac{K_{enkel1}}{sT+1} \cdot \frac{K_{enkel2}}{sT+1} = \frac{K\textsubscript{dubbel}}{(sT+1)^2}
\end{equation}

\\

\begin{equation}
\delta_{h1}(s) = \frac{C * K_{enkel1}}{sT+1}\frac{1}{s}
\end{equation}
\\
\begin{equation}
\delta_{u}(t) = C{\cdot}K_{enkel}(1-e^{(-\frac{t}{T})}) = 0,63K_{enkel}C 
\end{equation}


\begin{equation}
T = \frac{-t}{ln{0,37}} \approx t
\end{equation}

\\


\begin{equation}
\delta_{h2}(s) = \frac{K_{dubbel}}{(1+s)^2}*\frac{C}{s}
\end{equation}

\begin{equation}
\lim\limits_{t \to \infty}y(t) = \lim\limits_{s \to 0}sY(s)
\end{equation}
\\

\begin{equation}
K_{dubbel} = \frac{\delta_{h2}(s)}{C}
\end{equation}
\\


\begin{equation}
T_{i} = \frac{10}{\omega_{c,d}}
\end{equation}
\\


% ------------- Felanalys -----------------------


\section{Felanalys}


% ------------- Resultat--------------

\section{Resultat}

%--------figur-----------------------

%\begin{figure}[ht!]
%\centering
%\includegraphics[width=90mm]{}
%\caption{}
%\label{}
%\end{figure}


% --------------- Slutsats -------------------------

%\newpage
%\section{Slutsats}


% ------------ Källor -------------------------

\section{Källor}
\label{boken}
Glad, Torkel och Ljung, Lennart. 2006. \textit{Reglerteknik - Grundläggande teori}. Upplaga 4:10. Lund. Studentlitteratur AB.
\newpage



\end{document}

% Lite information om hur man arbetar med LaTeX
%-----------------------------------------------
%
% LaTeX-koden kan skrivas med en godtycklig editor.
% För att "kompilera" dokumentet används kommandot latex:
%    bergner@peppar:~/edu/sysprog/lab1> latex rapportmall.tex
% Resultatet blir ett antal filer, bl.a. en som heter rapportmall.dvi.
% Denna fil kan användas för att titta hur dokumentet egentligen ser
% ut med hjälp av programmet xdvi:
%    bergner@peppar:~/edu/sysprog/lab1> xdvi rapportmall.dvi &
% Du får då upp ett fönster som visar ditt dokument. Detta fönster
% kommer automatiskt att uppdateras då du ändrar och kompilerar om din
% LaTeX-kod. 
% När du anser att din rapport är färdig att skrivas ut använder man
% lämpligtvis kommandona dvips och lpr:
%    bergner@peppar:~/edu/sysprog/lab1> dvips -P ma436ps rapportmall.dvi
% Om man vill ha kvar PostScript-filen som dvips genererar kan man göra:
%    bergner@peppar:~/edu/sysprog/lab1> dvips -o rapport.ps rapportmall.dvi
%    bergner@peppar:~/edu/sysprog/lab1> lpr -P ma436ps rapport.ps
% OBS!!! För att innehållsförteckningen och eventuella referenser till
% tabeller och figurer garanterat ska stämma måste man köra latex 2ggr
% på sitt dokument efter att man har ändrat något.
%
%
% Lite information om saker man kan tänkas behöva i sitt arbete med LaTeX
%-------------------------------------------------------------------------
%
% FORMATTERA TEXT
%
% För att formattera text på lite olika sätt kan man använda följande LaTeX-
% kommandon:
%    \textbf{denna text kommer att vara i fetstil}
%    \emph{denna text är viktig (kursiv stil)}
%    \texttt{i denna text blir alla tecken lika breda, som med en skrivmaskin}
%    \textsf{denna text visas med ett typsnitt utan serifer}
%
%
% MATEMATISKA FORMLER
%
% För att typsätta matematiska formler kan man använda:
%    $f(x) = x^2 - 3$, vilket lägger in formeln i texten, eller
%    \begin{displaymath}
%        g(x) = \frac{\sin x}{x}
%    \end{displaymath}, vilket låter formeln visas centrerat på en egen rad
% Om du vill att en formel ska numreras byter du ut displaymath mot equation.
% Det finns massor med matematiska symboler, vilket gör att man behöver
% någon liten manual att titta i om man ska konstruera avancerade formler.
% Se slutet på filen för lite råd om var du kan hitta sådana.
%
%
% INFOGA FIGURER
%
% För att infoga en figur kan man göra på följande sätt:
%    \begin{figure}[htb]
%        \includegraphics[scale=0.5, angle=90]{exec_flow.eps}
%        \caption{Detta är bildtexten}
%        \label{EXECFLOW}
%    \end{figure}
% Om man vill referera till denna bild i texten skulle man då skriva enligt:
%    ...i figur \ref{EXECFLOW} kan man se att...
% Några små förklaringar till figurer:
%    [htb] = talar om hur latex ska försöka placera bilden (Here, Top, Bottom)
%            Om du använder [!h], innebär det Here!!!
%    scale = kan skala om bilden, om den är skalbar
%    angle = kan rotera bilden
%    exec_flow.eps = filnamnet på bilden. Notera att formatet .EPS används
% För att skapa figurer används lämpligtvis programmet xfig:
%    bergner@peppar:~/edu/sysprog/lab1> xfig &
% Rita (och spara ofta) tills du är klar. Välj sedan "Export" och exportera
% din figur till EPS-format.
% Om man vill kan man använda endast \includegraphics, men det är inte ofta
% man gör det.
%
%
% INFOGA TABELLER
%
% Om man vill skapa en tabell gör man på följande sätt:
%    \begin{table}[htb]
%        \begin{tabular}{|rlp{10cm}|}
%            \hline
%            13 & $17.26$ & En kommentar som kan sträcka sig över flera rader \\
%            \hline
%        \end{tabular}
%        \caption{Tabelltexten...}
%        \label{TBL:MINTABELL}
%    \end{table}
% Om man vill kan man endast använda raderna 2-6, dvs få en tabell utan text
% och nummer. Om man gör på detta vis kommer tabellen alltid att läggas på
% det ställe den skrivs i koden, dvs ungefär samma sak som [!h] -> Here!!!
% Några förklaringar:
%    l, r, c = vänsterjustera, högerjustera eller centrera kolumn
%    p{bredd} = skapa en vänsterjusterad kolumn med en viss bredd
%               kan innehålla flera rader text
%    | = en vertikal linje i tabellen
%    \hline = en horisontell linje i tabellen
%    & = kolumnseparator
%    \\ = radseparator
% Tänk på att tabeller oftast ser bättre ut med ganska få linjer.
%
%
% INFOGA KÄLLKOD ELLER UTDATA FRÅN TESTKÖRNINGAR
%
% Om man vill infoga källkod eller något annat liknande, t.ex. utdata från
% en testkörning är det bra om LaTeX återger utdatan korrekt, dvs en radbrytning
% betyder en radbrytning och 8 mellanslag på rad betyder 8 mellanslag på rad.
% För att åstadkomma detta används:
%    \begin{verbatim}
%        allt som skrivs här återges exakt, med skrivmaskinstypsnitt
%    \end{verbatim}
% Oftast finns det dock bättre verktyg för att skriva ut källkod. Exempel på
% sådana är a2ps, enscript och atp.
%
%
% ÄNDRA STORLEK PÅ TEXT
%
% Om du vill ändra storleken på ett stycke, t.ex. på din nyss infogade
% testkörning omger du stycket med \begin{STORLEK} \end{STORLEK}, där
% STORLEK är någon av:
%    tiny, scriptsize, footnotesize, small, normalsize, large, Large,
%    LARGE, huge, Huge
% Tänk på att inte mixtra för mycket med storlekar bara.
%
%
% SKAPA LISTOR AV OLIKA SLAG
%
% Det är ganska vanligt att man vill rada upp saker på något sätt. För att
% skapa punktlistor används:
%    \begin{itemize}
%        \item Detta är första punkten
%        \item Detta är andra punkten
%    \end{itemize}
% Om man istället vill ha en numrerad lista kan man använda enumerate istället
% för itemize. Listor kan användas i flera nivåer
%
%
% MER INFORMATION OM LaTeX
%
% Lite blandad information om LaTeX, länkar och annat hittar du på
% http://www.cs.umu.se/~bergner/latex.htm
% En del information om rapportskrivning hittar du på
% http://www.cs.umu.se/~bergner/rapport/
% Det finns massor med information om LaTeX på Internet. Ett litet urval:
% http://www.giss.nasa.gov/latex/
%     är en mycket välfylld sida om LaTeX
% http://wwwinfo.cern.ch/asdoc/WWW/essential/essential.html
%     är en manual som genererats utifrån ett LaTeX-dokument mha latex2html
% http://tex.loria.fr/english/
%     är ett fylligt arkiv av länkar till LaTeX-dokument på Internet
%
% Min personliga favorit är dock manualen "The Not So Short Introduction to
% LaTeX2e", som finns i DVI-format på ~bergner/LaTeX/lshort2e.dvi
% Där står i princip allt man behöver veta. Det är bara att använda xdvi och
% titta efter det du söker, vilket oftast finns där.
% Om du, precis som jag, vill kunna leka med många kommandon i LaTeX finns en
% "LaTeX Command Summary" på ~bergner/LaTeX/latexcmds.ps