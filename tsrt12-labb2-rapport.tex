%______________________________________________________
%
%   LaTeX-mall för nybörjare
%
%   Konstruerad av Marcus Bergner, bergner@cs.umu.se
%
%   Vid funderingar titta längst ned i denna fil,
%   eller skicka ett mail
%______________________________________________________
%

% lite inställningar
\documentclass[10pt, titlepage, oneside, a4paper]{article}
\usepackage[swedish]{babel}
\usepackage[T1]{fontenc}
\usepackage[utf8]{inputenc}
\usepackage{amssymb, graphicx, fancyheadings}
\addtolength{\textheight}{20mm}
\addtolength{\voffset}{-5mm}
\renewcommand{\sectionmark}[1]{\markleft{#1}}

% \Section ger mindre spillutrymme, använd dem om du vill
\newcommand{\Section}[1]{\section{#1}\vspace{-8pt}}
\newcommand{\Subsection}[1]{\vspace{-4pt}\subsection{#1}\vspace{-8pt}}
\newcommand{\Subsubsection}[1]{\vspace{-4pt}\subsubsection{#1}\vspace{-8pt}}
	
% appendices, \appitem och \appsubitem är för bilagor
\newcounter{appendixpage}

\newenvironment{appendices}{
	\setcounter{appendixpage}{\arabic{page}}
	\stepcounter{appendixpage}
}{
}

\newcommand{\appitem}[2]{
	\stepcounter{section}
	\addtocontents{toc}{\protect\contentsline{section}{\numberline{\Alph{section}}#1}{\arabic{appendixpage}}}
	\addtocounter{appendixpage}{#2}
}

\newcommand{\appsubitem}[2]{
	\stepcounter{subsection}
	\addtocontents{toc}{\protect\contentsline{subsection}{\numberline{\Alph{section}.\arabic{subsection}}#1}{\arabic{appendixpage}}}
	\addtocounter{appendixpage}{#2}
}

% Ändra de rader som behöver ändras
\def\inst{Systemvetenskap}
\def\typeofdoc{Laborationsrapport}
\def\course{Reglerteknik}
\def\pretitle{Laboration 2}
\def\title{Modellbaserad reglering av dubbeltankar}
\def\name{Peter Arvidsson, Sara Svensson}
%\def\username{c00abc}
%\def\email{\username{}@cs.umu.se}
\def\email{petarXXX@student.liu.se, sasv839@student.liu.se}
%\def\path{edu/KURS/lab1}
%\def\graders{ALLA HANDLEDARNAS NAMN}


% om du vill referera till katalogen där dina filer ligger kan du 
% använda \fullpath som kommer att vara "~username/edu..." o.s.v.
%\def\fullpath{\raisebox{1pt}{$\scriptstyle \sim$}\username/\path}


% Här börjar själva dokumentet
\begin{document}

	% skapar framsidan (om den inte duger: gör helt enkelt en egen)
	\begin{titlepage}
		\thispagestyle{empty}
		\begin{large}
			\begin{tabular}{@{}p{\textwidth}@{}}
				\textbf{Linköpings universitet \hfill \today} \\
				\textbf{Institutionen för \inst} \\
				\textbf{\typeofdoc} \\
			\end{tabular}
		\end{large}
		\vspace{10mm}
		\begin{center}
			\LARGE{\pretitle} \\
			\huge{\textbf{\course}}\\
			\vspace{10mm}
			\LARGE{\title} \\
			\vspace{15mm}
			\begin{large}
				\begin{tabular}{ll}
					\textbf{Namn} & \name \\
					\textbf{E-mail} & \texttt{\email} \\
					%\textbf{Sökväg} & \texttt{\fullpath} \\
				\end{tabular}
			\end{large}
			\vfill
			%\large{\textbf{Handledare}}\\
			\mbox{\large{\graders}}
		\end{center}
	\end{titlepage}


	% fixar sidfot
	\lfoot{\footnotesize{\name, \email}}
	\rfoot{\footnotesize{\today}}
	\lhead{\sc\footnotesize\title}
	\rhead{\nouppercase{\sc\footnotesize\leftmark}}
	\pagestyle{fancy}
	\renewcommand{\headrulewidth}{0.2pt}
	\renewcommand{\footrulewidth}{0.2pt}

	% skapar innehållsförteckning.
	% Tänk på att köra latex 2ggr för att uppdatera allt
	\pagenumbering{roman}
	\tableofcontents
	
	% och lägger in en sidbrytning
	\newpage

	\pagenumbering{arabic}

	% i Sverige har vi normalt inget indrag vid nytt stycke
	\setlength{\parindent}{0pt}
	% men däremot lite mellanrum
	\setlength{\parskip}{10pt}

	% lägger in rubrik (finns \section, men då får man mycket spillutrymme)
	\Section{Problemspecifikation}
		% \emph innebär emphasize, d.v.s. betona eller framhåll -> kursiv stil
		\emph{I detta avsnitt beskrivs laborationen, dels som en sammanfattning
			över dess syfte, men även en koppling till orginalspecifikationen
			tas upp.}

		% lägg in en underrubrik (\subsection -> spillutrymme)
		\Subsection{Problemsammanfattning}

		\Subsection{Orginalspecifikation}
			Specifikationen i sin helhet finns ...

			% nytt stycke skapas när du har en, eller flera, tomma rader
			någonstans i föregående stycke

	\Section{Åtkomst och användarhandledning}
		\emph{I detta avsnitt beskrivs de filer, med tillhörande sökvägar,
			som ingår i lösningen, samt korta beskrivningar om syftet med
			varje fil. Dessutom beskrivs hur man ska handskas med den
			implementerade lösningen till problemet, i form av instruktioner
			för hur körning och kompilering av lösningen sker.}
	
		\Subsection{Filer som ingår i lösningen}
			I katalogen \texttt{\fullpath} ...
			
		\Subsection{Kompilering och körning}
			

	\Section{Systembeskrivning}
		\emph{I detta avsnitt beskrivs systemet i sin helhet mer i detalj.
			Datastrukturer och annan intern representation som är central
			för uppgiftens lösning behandlas. Dessutom beskrivs de olika
			komponenternas relationer till varandra.}

		%Dessa rubriker behöver inte finnas med, men något i denna stil
		%bör det kanske vara
		\Subsection{Systemöversikt}

		\Subsection{Exekveringsflöde}
			
		\Subsection{Övriga systemdetaljer}

	\Section{Algoritmbeskrivning}
		\emph{I detta avsnitt beskrivs de algoritmer som anses som
			icke-triviala i uppgiften.}
		
		\Subsection{Algoritm 1}
			
		\Subsection{Algoritm 2}	
	
	\Section{Lösningens begränsningar}
		\emph{I detta avsnitt beskrivs alla begränsningar som lösningen av
			uppgiften innehåller. Detta innefattar även funna begränsningar
			som strider mot specifikationen.}
		
		SKRIV TEXT HÄR...

	\Section{Problem och reflektioner}
		\emph{Här presenteras egna tankar kring uppgiften som sådan samt de
			problem som uppstått under arbetets gång.}
		
		OCH HÄR..

	\Section{Testkörningar}
		\emph{I detta avsnitt visas ett antal tester som utförts med programmet.
			Samtliga tester är kommenterade.}
		
		GLÖM INTE ATT KOMMENTERA TESTER!
		
		\begin{footnotesize}
			\begin{verbatim}
				Utdata kan lämpligtvis återges så här, men tänk på att
				        kommentera dina tester. Lägg märke till att rad-
				brytningarna och mellanslagen bevaras!
			\end{verbatim}
		\end{footnotesize}
		

	% här börjar alla bilagor. Denna måste finnas med även om bara
	% bilagor anges i \begin{appendices} ... \end{appendices}
	\appendix

	\Section{Bilaga 1}
	\ldots{}ligger direkt i dokumentet

	% bilagor, t.ex. källkod. En tom extrasida kommer att skrivas ut för
	% att få alla sidnummer att stämma
	\begin{appendices}
		\appitem{Källkod}{0}
		\appsubitem{\texttt{minfil.c}}{2}
		\appsubitem{\texttt{minfil.h}}{1}
		\appitem{En bilaga på 3 sidor}{3}
	\end{appendices}

\end{document}


% Lite information om hur man arbetar med LaTeX
%-----------------------------------------------
%
% LaTeX-koden kan skrivas med en godtycklig editor.
% För att "kompilera" dokumentet används kommandot latex:
%    bergner@peppar:~/edu/sysprog/lab1> latex rapportmall.tex
% Resultatet blir ett antal filer, bl.a. en som heter rapportmall.dvi.
% Denna fil kan användas för att titta hur dokumentet egentligen ser
% ut med hjälp av programmet xdvi:
%    bergner@peppar:~/edu/sysprog/lab1> xdvi rapportmall.dvi &
% Du får då upp ett fönster som visar ditt dokument. Detta fönster
% kommer automatiskt att uppdateras då du ändrar och kompilerar om din
% LaTeX-kod. 
% När du anser att din rapport är färdig att skrivas ut använder man
% lämpligtvis kommandona dvips och lpr:
%    bergner@peppar:~/edu/sysprog/lab1> dvips -P ma436ps rapportmall.dvi
% Om man vill ha kvar PostScript-filen som dvips genererar kan man göra:
%    bergner@peppar:~/edu/sysprog/lab1> dvips -o rapport.ps rapportmall.dvi
%    bergner@peppar:~/edu/sysprog/lab1> lpr -P ma436ps rapport.ps
% OBS!!! För att innehållsförteckningen och eventuella referenser till
% tabeller och figurer garanterat ska stämma måste man köra latex 2ggr
% på sitt dokument efter att man har ändrat något.
%
%
% Lite information om saker man kan tänkas behöva i sitt arbete med LaTeX
%-------------------------------------------------------------------------
%
% FORMATTERA TEXT
%
% För att formattera text på lite olika sätt kan man använda följande LaTeX-
% kommandon:
%    \textbf{denna text kommer att vara i fetstil}
%    \emph{denna text är viktig (kursiv stil)}
%    \texttt{i denna text blir alla tecken lika breda, som med en skrivmaskin}
%    \textsf{denna text visas med ett typsnitt utan serifer}
%
%
% MATEMATISKA FORMLER
%
% För att typsätta matematiska formler kan man använda:
%    $f(x) = x^2 - 3$, vilket lägger in formeln i texten, eller
%    \begin{displaymath}
%        g(x) = \frac{\sin x}{x}
%    \end{displaymath}, vilket låter formeln visas centrerat på en egen rad
% Om du vill att en formel ska numreras byter du ut displaymath mot equation.
% Det finns massor med matematiska symboler, vilket gör att man behöver
% någon liten manual att titta i om man ska konstruera avancerade formler.
% Se slutet på filen för lite råd om var du kan hitta sådana.
%
%
% INFOGA FIGURER
%
% För att infoga en figur kan man göra på följande sätt:
%    \begin{figure}[htb]
%        \includegraphics[scale=0.5, angle=90]{exec_flow.eps}
%        \caption{Detta är bildtexten}
%        \label{EXECFLOW}
%    \end{figure}
% Om man vill referera till denna bild i texten skulle man då skriva enligt:
%    ...i figur \ref{EXECFLOW} kan man se att...
% Några små förklaringar till figurer:
%    [htb] = talar om hur latex ska försöka placera bilden (Here, Top, Bottom)
%            Om du använder [!h], innebär det Here!!!
%    scale = kan skala om bilden, om den är skalbar
%    angle = kan rotera bilden
%    exec_flow.eps = filnamnet på bilden. Notera att formatet .EPS används
% För att skapa figurer används lämpligtvis programmet xfig:
%    bergner@peppar:~/edu/sysprog/lab1> xfig &
% Rita (och spara ofta) tills du är klar. Välj sedan "Export" och exportera
% din figur till EPS-format.
% Om man vill kan man använda endast \includegraphics, men det är inte ofta
% man gör det.
%
%
% INFOGA TABELLER
%
% Om man vill skapa en tabell gör man på följande sätt:
%    \begin{table}[htb]
%        \begin{tabular}{|rlp{10cm}|}
%            \hline
%            13 & $17.26$ & En kommentar som kan sträcka sig över flera rader \\
%            \hline
%        \end{tabular}
%        \caption{Tabelltexten...}
%        \label{TBL:MINTABELL}
%    \end{table}
% Om man vill kan man endast använda raderna 2-6, dvs få en tabell utan text
% och nummer. Om man gör på detta vis kommer tabellen alltid att läggas på
% det ställe den skrivs i koden, dvs ungefär samma sak som [!h] -> Here!!!
% Några förklaringar:
%    l, r, c = vänsterjustera, högerjustera eller centrera kolumn
%    p{bredd} = skapa en vänsterjusterad kolumn med en viss bredd
%               kan innehålla flera rader text
%    | = en vertikal linje i tabellen
%    \hline = en horisontell linje i tabellen
%    & = kolumnseparator
%    \\ = radseparator
% Tänk på att tabeller oftast ser bättre ut med ganska få linjer.
%
%
% INFOGA KÄLLKOD ELLER UTDATA FRÅN TESTKÖRNINGAR
%
% Om man vill infoga källkod eller något annat liknande, t.ex. utdata från
% en testkörning är det bra om LaTeX återger utdatan korrekt, dvs en radbrytning
% betyder en radbrytning och 8 mellanslag på rad betyder 8 mellanslag på rad.
% För att åstadkomma detta används:
%    \begin{verbatim}
%        allt som skrivs här återges exakt, med skrivmaskinstypsnitt
%    \end{verbatim}
% Oftast finns det dock bättre verktyg för att skriva ut källkod. Exempel på
% sådana är a2ps, enscript och atp.
%
%
% ÄNDRA STORLEK PÅ TEXT
%
% Om du vill ändra storleken på ett stycke, t.ex. på din nyss infogade
% testkörning omger du stycket med \begin{STORLEK} \end{STORLEK}, där
% STORLEK är någon av:
%    tiny, scriptsize, footnotesize, small, normalsize, large, Large,
%    LARGE, huge, Huge
% Tänk på att inte mixtra för mycket med storlekar bara.
%
%
% SKAPA LISTOR AV OLIKA SLAG
%
% Det är ganska vanligt att man vill rada upp saker på något sätt. För att
% skapa punktlistor används:
%    \begin{itemize}
%        \item Detta är första punkten
%        \item Detta är andra punkten
%    \end{itemize}
% Om man istället vill ha en numrerad lista kan man använda enumerate istället
% för itemize. Listor kan användas i flera nivåer
%
%
% MER INFORMATION OM LaTeX
%
% Lite blandad information om LaTeX, länkar och annat hittar du på
% http://www.cs.umu.se/~bergner/latex.htm
% En del information om rapportskrivning hittar du på
% http://www.cs.umu.se/~bergner/rapport/
% Det finns massor med information om LaTeX på Internet. Ett litet urval:
% http://www.giss.nasa.gov/latex/
%     är en mycket välfylld sida om LaTeX
% http://wwwinfo.cern.ch/asdoc/WWW/essential/essential.html
%     är en manual som genererats utifrån ett LaTeX-dokument mha latex2html
% http://tex.loria.fr/english/
%     är ett fylligt arkiv av länkar till LaTeX-dokument på Internet
%
% Min personliga favorit är dock manualen "The Not So Short Introduction to
% LaTeX2e", som finns i DVI-format på ~bergner/LaTeX/lshort2e.dvi
% Där står i princip allt man behöver veta. Det är bara att använda xdvi och
% titta efter det du söker, vilket oftast finns där.
% Om du, precis som jag, vill kunna leka med många kommandon i LaTeX finns en
% "LaTeX Command Summary" på ~bergner/LaTeX/latexcmds.ps